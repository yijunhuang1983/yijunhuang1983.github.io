%%%%%%%%%%%%%%%%%%%%%%%%%%%%%%%%%%%%%%%%%
% Long Professional Curriculum Vitae
% LaTeX Template
% Version 1.1 (9/12/12)
%
% This template has been downloaded from:
% http://www.latextemplates.com
%
% Original author:
% Rensselaer Polytechnic Institute (http://www.rpi.edu/dept/arc/training/latex/resumes/)
%
% Important note:
% This template requires the res.cls file to be in the same directory as the
% .tex file. The res.cls file provides the resume style used for structuring the
% document.
%
%%%%%%%%%%%%%%%%%%%%%%%%%%%%%%%%%%%%%%%%%

%----------------------------------------------------------------------------------------
%	PACKAGES AND OTHER DOCUMENT CONFIGURATIONS
%----------------------------------------------------------------------------------------

\documentclass[11pt]{res} % Use the res.cls style, the font size can be changed to 11pt or 12pt here

\usepackage{helvet} % Default font is the helvetica postscript font
%\usepackage{newcent} % To change the default font to the new century schoolbook postscript font uncomment this line and comment the one above

\newsectionwidth{0pt} % Stops section indenting

\begin{document}

%----------------------------------------------------------------------------------------
%	YOUR NAME AND ADDRESS(ES) SECTION
%----------------------------------------------------------------------------------------

\name{Yijun Huang, Ph.D.\\ \\} % Your name at the top

% If you don't want one of the addresses, simply remove all the text in the first or second \address{} bracket

%\address{{Doctor of Engineering at Chinese Academy of Sciences} } % Your address 1

\address{Email: huangyj0@gmail.com \\ Address: 192 Lac Kine Dr., Rochester, NY 14618\\ Cell: (585) 285-0795} % Your address 2

%----------------------------------------------------------------------------------------

\begin{resume}

%----------------------------------------------------------------------------------------
%	OBJECTIVE SECTION
%----------------------------------------------------------------------------------------

\section{\centerline{OBJECTIVE}}

\vspace{8pt} % Gap between title and text

A data scientist position uitlizing data mining, machine learning, and optimization skills to build the next generation of data driven products. 

%----------------------------------------------------------------------------------------
%	EDUCATION SECTION
%----------------------------------------------------------------------------------------

\section{\centerline{EDUCATION}} 

\vspace{8pt} % Gap between title and text

{\sl Volunteer Research Assistant}, Machine Learning, Data Mining, Optimization \\
University of Rochester, United States \hfill 2014-Present

{\sl Volunteer Research Assistant}, Machine Learning, Optimization\\
University of Wisconsin-Madison, United States \hfill 2013-2014

{\sl Doctor of Engineering}, Mechanical and Electrical Engineering\\
Chinese Academy of Sciences, Shenyang Institute of Automation, China \hfill 2005-2010\\ 
Thesis - 3D Reconstruction from Multiple Views 
 
{\sl Doctor of Engineering: Course Study}\\ 
University of Science and Technology of China \hfill 2005-2006

{\sl Bachelor of Engineering}, Automation \\ 
Nankai University, China \hfill 2001-2005

%----------------------------------------------------------------------------------------
 
\vspace{0.2in} % Some whitespace between sections

%----------------------------------------------------------------------------------------
%	WORKING EXPERIENCE SECTION
%----------------------------------------------------------------------------------------

\section{\centerline{WORKING EXPERIENCES}} 

\vspace{8pt} % Gap between title and text

{\sl Volunteer Research Assistant}  \hfill 2014 - Present \\
University of Rochester, United States
\begin{itemize} \itemsep -2pt % Reduce space between items
\item Develop machine learning models and optimization algorithms for big data problems: feature selection, analyze social media data and medical longitudinal data for healthcare, active learning.
\item Develop AsynML: an asynchronous parallel algorithm package for solving some popular machine learning problems on multi-core / multi-socket systems.
\end{itemize}


{\sl Volunteer Research Assistant} \hfill 2013 - 2014 \\
University of Wisconsin-Madison, United States 
\begin{itemize} \itemsep -2pt % Reduce space between items
\item Develop an asynchronous parallel framework on multi-core / multi-socket system to implement optimization algorithms: deep learning, linear regression, SVM, logistic regression, linear system.
\end{itemize}

{\sl Senior Software Engineer} \hfill 2010 - 2013 \\
Tianjin Jinhang Institute of Computing Technology, China
\begin{itemize} \itemsep -2pt % Reduce space between items
\item Develop image/video applications in embedded system: path planing for aircraft, real-time infrared target recognition and tracking system, real-time communication system, electronic image stabilization system.
\end{itemize}

{\sl Research Assistant} \hfill 2006 - 2010 \\
Chinese Academy of Sciences, Shenyang Institute of Automation
\begin{itemize} \itemsep -2pt % Reduce space between items
\item 3D modeling / reconstruction, digital camera calibration, and digital visualization.
\end{itemize}


%----------------------------------------------------------------------------------------
%	PUBLICATIONS SECTION
%----------------------------------------------------------------------------------------

\vspace{0.2in} 
\section{\centerline{PUBLICATIONS}} 

\vspace{15pt} % Gap between title and text

\begin{itemize} \itemsep -2pt % Reduce space between items

\item H. Wang$^*$, \textbf{Y. Huang}$^*$, J. Liu, H. Huang, ``New Balanced Active Learning Model and Optimization Algorithm", AAAI, 2017. ($^*$ equal contribution. under review)
\item X. Lian, H. Zhang, C.-J. Hsieh, \textbf{Y. Huang}, and J. Liu, ``A Comprehensive Linear Speedup Analysis for Asynchronous Stochastic Parallel Optimization from Zeroth-Order to First-Order'', NIPS, 2016.
\item \textbf{Y. Huang} and J. Liu, ``Exclusive Sparsity Norm Minimization with Random Groups via Cone Projection", 2015. ArXiv:1510.07925 (under review).
\item \textbf{Y. Huang}, Q. Meng, J. Liu and S. Huang ``CHI: A Contemporaneous Health Index for Disease Monitoring using Longitudinal Data", 2016. (under review).
\item H. Yang, \textbf{Y. Huang}, L. Tran, J. Liu and S. Huang, ``On Benefits of Selection Diversity via Bilevel Exclusive Sparsity", CVPR, 2016.
\item X. Lian, \textbf{Y. Huang}, Y. Li, J. Liu, ``Asynchronous Parallel Stochastic Gradient for Nonconvex Optimization", NIPS, 2015.
\item J. Zhao, R. Xia, W. Liu, J. Xu and \textbf{Y. Huang}, ``Research on Volume Measurement Technology for Rail Tanker Based on Computer Vsion", International Conference on Mechatronics and Applied Mechanics, 2011. 
\item F. Yang, W. Liu and \textbf{Y. Huang}, ``A Method of Automatic Wheel Identify and Classify", Chinese Journal of Microcomputer Information, 2010. 
\item \textbf{Y. Huang} and W. Liu, ``Robust Estimation of the Fundamental Matrix Based on LTS and Bucketing", International Conference on Wavelet Analysis and Pattern Recognition, 2009.
\item \textbf{Y. Huang}, W. Liu and J. Zhao, ``Metric Reconstruction Based on Multifocal Tensors", IEEE International Conference on Intelligent Computing and Intelligent Systems, 2009. 
\item \textbf{Y. Huang}, W. Liu and J. Zhao, ``An Approach to Metric Reconstruction Based on Trifocal Tensor", Chinese Journal of Scientific Instrument, 2009. 
\item \textbf{Y. Huang} and W. Liu, ``A Method for Fundamental Matrix Estimation Using LQS", Journal of Image and Graphics, 2009. 
\end{itemize}

\vspace{0.2in} % Some whitespace between sections

%----------------------------------------------------------------------------------------
%	PATENS SECTION
%----------------------------------------------------------------------------------------


\section{\centerline{PATENTS}} 

\vspace{15pt} % Gap between title and text

\begin{itemize} \itemsep -2pt % Reduce space between items
\item \textbf{Y. Huang}, W. Liu and J. Zhao, ``An Approach to Metric Reconstruction Based on Trifocal Tensor", Cn101750029a, 2009.
\item J. Zhao, R. Xia, W. Liu, \textbf{Y. Huang}, ``An Approach on volume measurement technology for rail tanker based on computer vision", Cn101629805, 2010. 
\end{itemize}

%----------------------------------------------------------------------------------------
 
\vspace{0.2in} % Some whitespace between sections

%----------------------------------------------------------------------------------------
%	PROJECTS SECTION
%----------------------------------------------------------------------------------------

\section{\centerline{PROJECTS}} 

\vspace{8pt} % Gap between title and text

{\sl Machine learning and optimization algorithms development for big data problems}  \hfill 07/2013 - Now %\\[2pt]


Project 1: Work with multiple big and complex data sources (from soucial media, health record, business survery, biomedical data, etc), utilize machine learning techniques to build predictive models to meet the pratical needs, and solve these models by designing efficient optimization methods.
\begin{itemize} \itemsep -2pt % Reduce space between items
\item Develop two kinds of sparse feature selection methods: a bilevel exclusive sparsity algorithm which is to pursue the diversity by restricting the numbers of important features in the overall scale and in each feature group; an exclusive sparsity norm minimization method which is to pursue a sparse solution by making the importmant features evenly distributed in different feature groups.
\item Develop a range regression algorithm for ordinal labeling problems.
\item Develop a predict method for contemporaneous patient risk monitoring by combining longitudinal data that reflect the degeneration of the health condition.
\item Analyze Twittes and implement a label propagation method to predict users' health statuses.
\end{itemize} 



\vspace{-8pt} % Reduce space between positions at the same organization

Project 2: Develop an asynchronous parallel software framework for multi-core / multi-socket server to implement parallel optimization algorithms / solvers for big data problems in machine learning and scientific computing.
\begin{itemize} \itemsep -2pt % Reduce space between items
\item Build an unlocked asynchronous parallel software framework to manage distributed storage and distributed processing on the multi-core / multi-socket server (Intel Core / Intel Xeon), based on NUMA and POSIX Pthread libraries. 
\item Implement some popular asynchronous parallel optimization algorithms for big data problems: deep learning, linear system, SVM, LASSO, Logistic regression, etc. 
\end{itemize}


%%%%%%%%%%%%%%%%%%%%%%%%%%%%%%%%%%%%%%%%%%%%%%%

{\sl Embedded system development}  \hfill 06/2010 - 05/2013 %\\[2pt]

Project 3: Design a PCIe device for efficient path planning computing. 
\begin{itemize} \itemsep -2pt 
\item Designed and validated multiple image processing algorithms for path planning on PC.
\item Implemented these algorithms on FPGA  (mainly Xilinx Virtex-6) in Verilog.
\item Developed parallel implementations of these algorithms on Nvidia GPU for the purpose of comparison to the FPGA implementations.
\end{itemize}

\vspace{-8pt}

Project 4: Build a real-time infrared target recognition and tracking device for  aerocraft navigation.
\begin{itemize} \itemsep -2pt 
\item Designed multiple real-time infrared target recognition and tracking algorithms.
\item Implemented these algorithms on DSP  (TI TM320c64x, TI TM320c67x, ADSP210).
\end{itemize}

\vspace{-8pt}

Project 5: Develop real-time embedded control/communication devices for aviation applications.

\begin{itemize} \itemsep -2pt 
\item Developed embedded softwares on DSP/ARM for: interfacing with devices (SPI, UART, 1553 bus, and Ethernet  interface); testing hardware models; and implementing protocol stacks (e.g., UDP/IP).

\end{itemize}

\vspace{-8pt}

Project 6: Develop embedded EIS (Electronic Image Stabilization) system.

\begin{itemize} \itemsep -2pt 
\item Developed multiple efficient image deblurring algorithms to compensate for video device shake.
\item  Implemented these algorithms on DSP/ARM.
\end{itemize}


%%%%%%%%%%%%%%%%%%%%%%%%%%%%%%%%%%%%%%%%%%%%%%%

{\sl Graphics and Computer Vision}  \hfill 07/2006  - 06/2010 %\\[2pt]

Project 7: Develop a 3D modeling system to reconstruct and visualize the target object from an image sequence.
\begin{itemize} \itemsep -2pt 
\item Proposed an algorithm to obtain a 3D digital model of the target object from an image sequence (This algorithm improves the traditional Space Carving methods);
\item Proposed an efficient method for constructing the 3D visual hull.
\end{itemize}

\vspace{-8pt}

Project 8: Develop an image-based volume  measurement system for rail tankers.
\begin{itemize} \itemsep -2pt 
\item Proposed a high accuracy 3D reconstruction algorithm for our applications;
\item Calibrated cameras;
\item Proposed a uniform reconstruction framework based on multifocal tensors.
\end{itemize}


%----------------------------------------------------------------------------------------

\vspace{0.2in} % Some whitespace between sections

%----------------------------------------------------------------------------------------
%	HONORS SECTION
%----------------------------------------------------------------------------------------

\section{\centerline{HONORS}} 

%\vspace{-5pt} % Reduce space between section title and contents

\begin{itemize}
\item Rank 1 (1/200+) of National Higher Education Entrance Examination, Tianjin 32rd High School   \hfill  2001 
\item Outstanding Student Scholarship, Nankai University   \hfill  2005
\end{itemize}


%----------------------------------------------------------------------------------------
\vspace{0.2in} % Some whitespace between sections

%----------------------------------------------------------------------------------------
%	SKILLS SECTION
%----------------------------------------------------------------------------------------

\section{\centerline{SKILLS}}

\vspace{8pt} % Gap between title and text

Solid background and rich experiences of data analytics, machine learning, optimization, computer architecture, image processing and graphics.

Programming languages:
proficient in C$\backslash$C++ (10 years +), matlab (7 years +) and R \& Python (2 years +); experienced in Java and Verilog.

Development Experiences: 
POSIX Multi-threads Programming (3 years +), Hadoop \& Spark (1 year+), TI's DSP and ADI's ADSP (2 years +), ARM (2 years +), Xilinx FPGA (2 years+), NVIDIA GPU (1 year +), OpenCV (3 years+), OpenGL (2 years+) and GUI  (Qt, MFC) (3 years+).

Others:
experienced in SQL, OpenMP, Communication protocols (1553, TCP, UDP, UART).

Languages: Mandarin and English.

%----------------------------------------------------------------------------------------

\vspace{0.2in} % Some whitespace between sections



%----------------------------------------------------------------------------------------





%----------------------------------------------------------------------------------------


\end{resume} 
\end{document}
